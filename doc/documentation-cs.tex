
\documentclass[a4paper, 12pt]{article}
\usepackage[utf8]{inputenc}
\usepackage[czech]{babel}
% \usepackage{graphicx}
% \graphicspath{ {./} }

\title{Simulace Brownova pohybu}
\author{Richard Hartmann}

\begin{document}
    \maketitle
    \section*{Popis}
    Program simuluje pohyb částic(např. vody nebo vzduchu), které narážejí do větší částice(např. pylového zrnka nebo prachové částice) a způsobují její náhodný pohyb. Simulace se odehrává v izolované soustavě.
    \newline
    Každá častice má svou pozici, poloměr, hmotnost a vektor reprezentující její rychlost a směr.
    \newline
    Nejdříve si zjistíme, které částice budou kolidovat jako první. Pak se přesuneme o čas, za který mají kolidovat,  a náležitě přemístíme všechny častice. V tento okamžik přepočítáme nové pohybové vektory kolidujících částic(tak aby se nezměnila celková energie a hybnost před a po kolizi). Tento postup opakujeme.
    \newline
    Pro zobrazení simulace v reálném čase si čas každé následující kolize rozdělíme na předem dané úseky. Po uplynutí každého úseku celou soustavu vykreslíme na plochu.
    \section*{Zdroje}
    https://en.wikipedia.org/wiki/Brownian\_motion

\end{document}